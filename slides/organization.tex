\documentclass{beamer}
%
% Choose how your presentation looks.
%
% For more themes, color themes and font themes, see:
% http://deic.uab.es/~iblanes/beamer_gallery/index_by_theme.html
%
\mode<presentation>
{
  \usetheme{default}      % or try Darmstadt, Madrid, Warsaw, ...
  \usecolortheme{default} % or try albatross, beaver, crane, ...
  \usefonttheme{default}  % or try serif, structurebold, ...
  \setbeamertemplate{navigation symbols}{}
  \setbeamertemplate{caption}[numbered]
  \setbeamertemplate{footline}[frame number]
  \setbeamertemplate{itemize items}[circle]
}

\usepackage[english]{babel}
\usepackage[utf8]{inputenc}
\usepackage[T1]{fontenc}
\usepackage{tabularx}

\def\code#1{\texttt{#1}}

\title[FPV revision course]{Functional Programming and Verification \\ revision course}
\author{Jonas Hübotter}

\begin{document}

\begin{frame}
  \titlepage
\end{frame}

% Uncomment these lines for an automatically generated outline.
% \begin{frame}{Outline}
%  \tableofcontents
% \end{frame}

\begin{frame}{Organization}

\begin{tabularx}{\textwidth}{lX}
    Wednesday, June 17th &
        \begin{tabular}[t]{ll}
        recursion, & \\
        list comprehensions, & \\
        higher-order functions & \\
        \end{tabular} \\
    Wednesday, June 24th &
        \begin{tabular}[t]{ll}
        algebraic data types, & \\
        type classes, & \\
        abstract data types, & \\
        type inference & \\
        \end{tabular} \\
    Friday, June 26th &
        \begin{tabular}[t]{ll}
        automated theorem proving & \\
        \end{tabular} \\
    Wednesday, July 1st &
        \begin{tabular}[t]{ll}
        IO, & \\
        evaluation/reduction & \\
        \end{tabular} \\
\end{tabularx}

\end{frame}

\begin{frame}{Schedule}

\begin{block}{Session 1}
\begin{itemize}
    \item Haskell fundamentals
    \item recursion, guards, pattern matching
    \item list comprehensions
    \item QuickCheck
    \item polymorphism
    \item currying, partial application
    \item higher-order functions (incl. \code{fold})
\end{itemize}
\end{block}

\begin{block}{Session 2}
\begin{itemize}
    \item type classes
    \item algebraic data types (incl. \code{Maybe})
    \item modules, abstract data types
    \item type inference
\end{itemize}
\end{block}

\end{frame}

\begin{frame}{Schedule}

\begin{block}{Session 3}
\begin{itemize}
    \item structural induction
    \item case analysis
    \item extensionality
    \item computation induction
\end{itemize}
\end{block}

\begin{block}{Session 4}
\begin{itemize}
    \item correctness
    \item I/O
    \item lazy evaluation, infinite data structures
    \item complexity and optimization
\end{itemize}
\end{block}

\end{frame}

\begin{frame}{Structure}

\begin{enumerate}
    \item I give a brief introduction to a topic
    \item we go over an example problem together
    \item you work on problems
    \item we compare results
    \item I provide additional practice problems and further references
\end{enumerate}

\vskip 1cm
\pause

Important for you:
\begin{itemize}
    \item ask questions!
    \item let me know when you want to spend more time on a topic
\end{itemize}

\vskip 1cm
\pause

Slides, problems, and solutions can be found on GitHub:
\url{https://github.com/jonhue/teaching-fpv-rev}

\end{frame}

\end{document}
