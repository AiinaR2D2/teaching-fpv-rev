\documentclass{article}
\usepackage[utf8]{inputenc}
\usepackage[english]{babel}
\usepackage{amsmath}

\usepackage{biblatex}
\addbibresource{sources.bib}

\evensidemargin=0in
\oddsidemargin=0in
\textwidth=6.5in
\textheight=9.0in

\title{Functional Programming and Verification \\ revision course \\ Types}
\author{Jonas Hübotter}
\date{June 24, 2020}

\begin{document}

\maketitle

\section{Exercises}

\subsection{Review}
\begin{enumerate}
\item {[endterm 2015, 3]}
\item {[endterm 2014]} A \textit{Collatz} sequence is a special sequence of natural numbers. The element $c_{k+1}$ is obtained from the previous element $c_k$ as follows:
\[
c_{k+1} =
\begin{cases}
  \frac{c_k}{2} & \text{if $c_k$ even} \\
  3 \cdot c_k + 1 & \text{if $c_k$ odd}
\end{cases}
\]
The first element $c_0$ can be any natural number $n > 0$. The sequence ends when the number $1$ is reached. \par
\textit{Example:} let $c_0 = 6$. The entire sequence then is $6,3,10,5,16,8,4,2,1$.
\begin{enumerate}
\item Define a function \verb|collatz :: Integer -> [Integer]| that given an initial value returns the Collatz sequence as a list.
\item Implement the function \verb|unfold :: (a -> Maybe a) -> a -> [a]|. For a call \verb|unfold f a| the function $f$ should be repeatedly applied to $a$ and stop with a result of \verb|Nothing|. The return value is a list of all intermediate results, including the initial value of $a$. \par
\textit{Example:} For $f a_0 =$ \verb|Just| $a_1$, $f a_1 =$ \verb|Just| $a_2$, $f a_n =$ \verb|Nothing| we have \verb|unfold| $f a_0 = [a_0, a_1, \dots, a_n]$.
\item Find a function \verb|next| such that \verb|collatz n = unfold next n| for $n > 0$.
\end{enumerate}
\end{enumerate}

\section{Homework}

\printbibliography

\end{document}
