\documentclass{article}
\usepackage[utf8]{inputenc}
\usepackage[english]{babel}
\usepackage{alltt}
\usepackage{amsmath}
\usepackage{csquotes}

\def\code#1{\texttt{#1}}

\usepackage{biblatex}
\addbibresource{sources.bib}

\evensidemargin=0in
\oddsidemargin=0in
\textwidth=6.5in
\textheight=9.0in

\title{Types}
\author{Jonas Hübotter}

\begin{document}

\maketitle

\section{Exercises}

\subsection{Type Classes}
\begin{enumerate}
    \item {[endterm 2020]} We define a typeclass of integer containers as follows:
        \begin{verbatim}
    class IntContainer c where
      -- the empty container
      empty :: c
      -- insert an integer into a container
      insert :: Integer -> c -> c
        \end{verbatim}
        Moreover, we define an extension of integer containers called \code{IntCollection} as follows:
        \begin{verbatim}
    class IntContainer c => IntCollection c where
      -- the number of integers in the collection
      size :: c -> Integer
      -- True if and only if the integer is a member of the collection
      member :: Integer -> c -> Bool
      -- extracts the smallest number in the collection
      -- if such a number exists.
      extractMin :: c -> Maybe Integer
      -- "update f c" applies f to every element e of c.
      -- If "f c" returns Nothing, the element is deleted;
      -- otherwise, the new value is stored in place of e.
      update :: (Integer -> Maybe Integer) -> c -> c
      -- "partition p c" creates two collections (c1 , c2) such that
      -- c1 contains exactly those elements of c satisfying p and
      -- c2 contains exactly those elements of c not satisfying p.
      partition :: (Integer -> Bool) -> c -> (c,c)
        \end{verbatim}
        Assume there is a type \code{data C} with a corresponding \code{IntContainer} instance. Moreover, assume you are given the following function:
        \begin{verbatim}
    -- "fold f acc c" folds the function f along c (in no particular order)
    -- using the start accumulator acc.
    fold :: (Integer -> b -> b) -> b -> C -> b
        \end{verbatim}
        Define an instance \code{IntCollection C}.
\end{enumerate}

\subsection{Algebraic Data Types}
\begin{enumerate}
    \item {[endterm 2014]} A \textit{Collatz} sequence is a special sequence of natural numbers. The element $c_{k+1}$ is obtained from the previous element $c_k$ as follows:
    \[
    c_{k+1} =
    \begin{cases}
      \frac{c_k}{2} & \text{if $c_k$ even} \\
      3 \cdot c_k + 1 & \text{if $c_k$ odd}
    \end{cases}
    \]
    The first element $c_0$ can be any natural number $n > 0$. The sequence ends when the number $1$ is reached. \par
    \textit{Example:} let $c_0 = 6$. The entire sequence then is $6,3,10,5,16,8,4,2,1$.
    \begin{enumerate}
    \item Define a function \verb|collatz :: Integer -> [Integer]| that given an initial value returns the Collatz sequence as a list.
    \item Implement the function \verb|unfold :: (a -> Maybe a) -> a -> [a]|. For a call \verb|unfold f a| the function $f$ should be repeatedly applied to $a$ and stop with a result of \verb|Nothing|. The return value is a list of all intermediate results, including the initial value of $a$. \par
    \textit{Example:} For $f a_0 =$ \verb|Just| $a_1$, $f a_1 =$ \verb|Just| $a_2$, $f a_n =$ \verb|Nothing| we have \verb|unfold| $f a_0 = [a_0, a_1, \dots, a_n]$.
    \item Find a function \verb|next| such that \verb|collatz n = unfold next n| for $n > 0$.
    \end{enumerate}

    \item {[sheet 8]} Trees.
        \begin{enumerate}
            \item In a binary tree, we can only descend to the left or to the right. Define a data type \code{Tree} with constructors \code{Leaf} and \code{Node} for binary trees.
            \item Implement a function \code{sumTree :: Num a => Tree a -> a} that returns the sum of all values in a tree.
            \item Implement a function \code{cut :: Tree a -> Integer -> Tree a} that cuts off a tree after a given height.
            \item Implement a function \code{foldTree :: (a -> b -> b) -> b -> Tree a -> b} that folds a function over a tree. The fold should process the right children of a node first, then the node itself, and lastly the left children.
            \item Use \code{foldTree} to implement a function \code{inorder :: Tree a -> [a]} that returns all elements of a tree in left to right order.
            \item Use \code{foldTree} to implement a function \code{findAll :: (a -> Bool) -> Tree a -> [a]} that returns all elements of a tree that satisfy a given predicate.
        \end{enumerate}

    \item {[endterm 2020]} We define the following types:
        \begin{itemize}
            \item An \textit{atom} is either \code{F} (falsity), \code{T} (truth), or a variable:
                \begin{verbatim}
    type Name = String
    data Atom = F | T | V Name deriving (Eq, Show)
                \end{verbatim}
            \item A \textit{conjunction} is an atom or the conjunction of two conjunctions:
                \begin{verbatim}
    data Conj = A Atom | Conj :&: Conj deriving (Eq, Show)
                \end{verbatim}
        \end{itemize}
        \begin{enumerate}
            \item Write a function \code{contains :: Conj -> Atom -> Bool} such that \code{contains c a} returns \code{True} if and only if \code{a} occurs in \code{c}.
            \item Write a function \code{implConj :: Conj -> Conj -> Bool} such that \code{implConj c1 c2} returns \code{True} if and only if conjunction \code{c1} logically implies conjunction \code{c2}. For example:
                \begin{verbatim}
    A F `implConj` c = True -- for any conjunction c
    c `implConj` A T = True -- for any conjunction c
    A (V "v") `implConj` A (V "v") = True
    A (V "v") `implConj` A (V "v") :&: A (V "w") = False
    A (V "w") :&: A (V "v") `implConj` A (V "v") :&: A (V "w") = True
                \end{verbatim}
        \end{enumerate}

    \item {[sheet 12]} We consider a simple programming language, namely $\lambda$-calculus. It is the basis for most functional programming languages including Haskell. A program in $\lambda$-calculus is built up from terms, which are either
        \begin{itemize}
            \item just a variable, e.g. $x$,
            \item an anonymous function defintion $(\lambda x.\ T)$, which binds a variable $x$ in term $T$, or
            \item a function application $T\ U$ where both $T$ and $U$ are terms themselves.
        \end{itemize}
        Note that bound variable names are interchangeable whereas this is not the case for free variables. For example, the terms $(\lambda x.\ x\ y)$ and $(\lambda z.\ z\ y)$ are equal while the terms $(\lambda x.\ x\ y)$ and $(\lambda x.\ x\ z)$ are not equal. Start by defining a datatype \code{Term} in Haskell that models the $\lambda$-calculus using strings for variables names. On this datatype, implement the following functions:
        \begin{enumerate}
            \item Instantiate \code{Show} for \code{Term} by representing variables as just their name, $\lambda$-abstractions as \code{(\textbackslash x -> T)} like in Haskell and use spaces for function application.
            \item Define \code{freeVars :: Term -> [String]} which collects all free variables in a term, i.e. variables that are not bound by an enclosing $\lambda$-abstraction.
            \item Implement a function \code{substVar :: String -> Term -> Term -> Term} which, when called with \code{substVar x r t}, replaces all free occurences of \code{x} in \code{t} by the term \code{r}. \textit{Important:} the function \code{substVar} makes a key assumption about the terms \code{t} and \code{r}. To find out what the assumption is, think about what happens when substituting $x$ with the variable $y$ in the equivalent terms $(\lambda y.\ x\ y)$ and $(\lambda z.\ x\ z)$.
        \end{enumerate}

    \item {[sheet 11]} In this exercise we will work with the datatype \code{Either} from the Prelude, which is defined as follows:
        \begin{verbatim}
    data Either a b = Left a | Right b
        \end{verbatim}
        As its name suggests, this type is useful when you have a value which can either be of type \code{a} or type \code{b}. You might for instance use \code{Either} for a function that parses integers: The function could have the type \code{String -> Either String Integer} where a successful parse returns an integer as a \code{Right} value, whereas an input that cannot be parsed returns an error message as a \code{Left} value. \par
        Now to the task at hand: It can sometimes be interesting to determine an implementation for a given type signature. For example, the signature \code{Either a b -> Either b a} has the following implementation:
        \begin{verbatim}
    f :: Either a b -> Either b a
    f (Left x) = Right x
    f (Right x) = Left x
        \end{verbatim}
        Your task is to write total functions that implement the following signatures:
        \begin{verbatim}
    g :: (a -> a') -> (b -> b') -> Either a b -> Either a' b'
    h :: (a -> Either a b) -> a -> b
        \end{verbatim}
        Your functions may not throw exceptions or be \code{undefined} and if possible they should terminate on all inputs.
\end{enumerate}

\subsection{Abstract Data Types}
\begin{enumerate}
    \item {[sheet 12]} In this exercise you will model \textit{vectors}, which are essentially resizable arrays. By convention, we will index the cells of our vectors beginning with $0$. \par
        Define a module \code{Vector} that only exports a type \code{Vector a} and the following functions:
        \begin{verbatim}
    newtype Vector a = ...
    newVector :: Int -> Vector a
    size :: Vector a -> Int
    capacity :: Vector a -> Int
    resize :: Vector a -> Int -> Vector a
    set :: Vector a -> a -> Int -> Maybe (Vector a)
    get :: Vector a -> Int -> Maybe a
        \end{verbatim}
        \code{newVector n} should return an empty vector of size $n$. \code{size} returns the size of the vector, whereas \code{capacity} returns the number of empty cells in the vector. \par
        \code{resize} changes the size of the vector by either adding new empty cells or by truncating the cells with the highest indices. \par
        \code{set v x i} should set the cell at index \code{i} to \code{x}. If \code{i} is not a valid index, the function should return \code{Nothing}. \par
        \code{get v i} returns the element in cell \code{i}. If that cell is empty or if \code{i} is not a valid index, it should return \code{Nothing}.
\end{enumerate}

\subsection{Type inference}
\begin{enumerate}
    \item {[endterm 2014]} Why is it often better to write \code{null $xs$} instead of \code{$xs$ == []}?

    \item {[sheet 11]} Use the algorithm from the lecture to determine the most general types of the following definitions:
        \begin{enumerate}
            \item \code{ffoldl = foldl . foldl} (where \code{foldl :: (a -> b -> a) -> a -> [b] -> a})
            \item \code{f x y = y : map (++x) y}
        \end{enumerate}

    \item {[endterm 2020]} Give a brief justification why these expressions do not type check.
        \begin{enumerate}
            \item \code{if f x then x else "error"} (where \code{f :: (a -> Bool)} and \code{x :: Int})
            \item \code{1 : 2 : f x} (where \code{f :: (a -> String)} and \code{x :: Int})
        \end{enumerate}
\end{enumerate}

\section{Homework}

\subsection{Type Classes}
\begin{enumerate}
    \item {[sheet 15]} Consider the classes \code{Semigroup} and \code{Monoid}:
        \begin{verbatim}
    class Semigroup a where
      (<>) :: a -> a -> a

    class Semigroup m => Monoid m where
      mempty :: m
        \end{verbatim}
        We define the type of pairs as follows:
        \begin{verbatim}
    data Pair a = Pair a a
        \end{verbatim}
        Your task is to write instances of \code{Semigroup} and \code{Monoid} for \code{Pair}. For \code{Semigroup}, you have to implement the operation \code{<>} which should combine two pairs by applying \code{<>} componentwise. Make sure the operation is associative:
        \begin{verbatim}
    Pair a b <> (Pair c d <> Pair e f) =
    (Pair a b <> Pair c d) <> Pair e f
        \end{verbatim}
        \code{Monoid} requires you to give a neutral element \code{mempty} with respect to \code{<>}, i.e:
        \begin{verbatim}
    Pair a b <> mempty = mempty <> Pair a b = Pair a b
        \end{verbatim}
\end{enumerate}

\subsection{Algebraic Data Types}
\begin{enumerate}
    \item {[endterm 2015]}
        \begin{enumerate}
            \item Define the functions \code{safeHead :: [a] -> Maybe a} and \code{safeLast :: [a] -> Maybe a} that, if possible, return the first and last element of a list respectively (otherwise \code{Nothing}). Do not use any predefined functions.
            \item Define \code{safeHead} and \code{safeLast} again. Use \code{foldr}, but none of the following techniques: list comprehensions, recursion, pattern matching on lists and predefined functions (except \code{foldr}). \par
            \textit{Note:} \code{foldr} is defined as follows:
            \begin{verbatim}
        foldr :: (a -> b -> b) -> b -> [a] -> b
        foldr f a [] = a
        foldr f a (x:xs) = f x (foldr f a xs)
            \end{verbatim}
            \item Define the function \code{select :: Eq a => a -> [(a,b)] -> [b]} that given a value $x$ and a list of pairs $(y,z)$ returns all values $z$ such that $y = x$. Use \code{map} and \code{filter}, but none of the following techniques: list comprehensions, recursion and pattern matching on lists. \par
            \textit{Examples:}
            \begin{verbatim}
        select ’ a ’ [(’b’, 1), (’a’, 3), (’c’, 4), (’a’, 2)] == [3, 2]
        select ’ d ’ [(’b’, 1), (’a’, 3), (’c’, 4), (’a’, 2)] == []
            \end{verbatim}
        \end{enumerate}

    \item {[endterm 2014]} \code{Node} is a datatype representing nodes in a file system:
        \begin{verbatim}
    data Node = File String | Dir String [Node]
        \end{verbatim}
        A node is either a \textit{normal file} (\code{File}) or a \textit{directory} (\code{Dir}). Nodes have a name (of type \code{String}). Additionally, directories contain a list of nodes. An entire file system can be represented by a list of nodes:
        \begin{verbatim}
    type FileSys = [Node]
        \end{verbatim}
        Example:
        \begin{verbatim}
    myFileSys =
      [Dir "Applications"
         [Dir "Aquamacs.app" [File "Info.plist"],
          Dir "Scribus.app" [File "Info.plist"]],
       Dir "Library"
         [Dir "Automator" [],
          Dir "Logs" []],
       Dir "Users"
         [Dir "smith"
            [Dir "bugs" [File "Scratch.hs"],
             File "Scratch.hs"]]]
        \end{verbatim}
        Implement a function \code{removeFiles :: String -> FileSys -> FileSys} that deletes all normal files with the given name.

    \item {[sheet 9]} An \textit{atom} is either a variable labelled by a string or the value \code{T} representing truth. A \textit{literal} is a positive or negative atom. Finally, a \textit{formula} is either a literal, a conjunction of formulas, or a disjunction of formulas.
        \begin{enumerate}
            \item Define data types for atoms, literals, and formulas. Make your definitions derive instances for \code{Eq} and \code{Show}.
            \item Define values \code{top :: Literal} and \code{bottom :: Literal} representing truth and falsity, respectively.
            \item A \textit{clause} is a disjunction of literals. A formula is in \textit{conjunctive normal form} (CNF) if it is a conjunction of clauses. We define the corresponding types
                \begin{verbatim}
    type Clause = [Literal]
    type CNF = [Clause]
                \end{verbatim}
                By convention, an empty clause corresponds to falsity since it is the neutral element of the disjunction operator. Similarly, \code{[] :: CNF} corresponds to truth - as it is the neutral element of the conjunction operator. \par
                Define a function \code{conjToForm :: CNF -> Formula} that transforms a formula given as a list of clauses to a formula as encoded in subtask (a). Examples:
                \begin{verbatim}
    conjToForm [] = L top
    conjToForm [[]] = L bottom
    conjToForm [[] , []] = L bottom :&: L bottom
    conjToForm [[Pos $ Var "v1", Neg $ Var "v2"], [Neg $ Var "v1", top]]
      = (L (Pos (Var "v1")) :|: L (Neg (Var "v2"))) :&:
        (L (Neg (Var "v1")) :|: L top)
                \end{verbatim}
                \textit{Hint:} It might be useful to first define a function of type \code{Clause -> Formula}.
            \item Given the type of \textit{valuations} \code{type Valuation = [(Name,Bool)]} write a function \code{substConj :: Valuation -> CNF -> CNF} that replaces the variables of a formula by the values as specified in the passed valuation. Examples:
                \begin{verbatim}
    substConj [("v", True)] [[Neg $ Var "v"], [Pos $ Var "w"]]
      = [[bottom], [Pos (Var "w")]]
    substConj [("v", False)] [[Neg $ Var "v"], [Pos $ Var "w"]]
      = [[top], [Pos (Var "w")]]
    substConj [("v", True), ("w", False)]
      [[Neg $ Var "v"], [Pos $ Var "w", Neg $ Var "w"]]
      = [[bottom], [bottom, top]]
                \end{verbatim}
            \item Write a function \code{simpConj :: CNF -> CNF} that simplifies a formula $F$ in CNF in the following way:
                \begin{enumerate}
                    \item For every clause $C$ in $F$ containing \code{top}, simplify $C$ to \code{[]top]}.
                    \item For every clause $C$ in $F$, remove every occurence of \code{bottom}.
                    \item Remove every clause $C$ in $F$ that has been simplified to \code{[top]}.
                    \item If $F$ contains a clause $C$ that has been simplified to the empty clause, simplify $F$ to \code{[[]]}.
                \end{enumerate}
                Examples:
                \begin{verbatim}
simpConj [[top], [top, Pos $ Var "v"]] = []
simpConj [[bottom], [Neg $ Var "v"]] = [[]]
simpConj [[Neg $ Var "v"], [bottom, Neg $ Var "v"]]
  = [[Neg (Var "v")], [Neg (Var "v")]]
simpConj [[Pos $ Var "v", bottom], [Pos $ Var "v", top]]
  = [[Pos (Var "v")]]
simpConj [[Neg $ Var "v"], [Neg $ Var "v"],
          [Pos $ Var "w", Neg $ Var "w"]]
  = [[Neg (Var "v")], [Neg (Var "v")], [Pos (Var "w"), Neg (Var "w")]]
                \end{verbatim}
            \item Write a function \code{cnf :: Formula -> CNF} that transforms a formula into CNF. Note that if $\phi = \phi_1 \land \dots \land \phi_n$ and $\psi = \psi_1 \land \dots \land \psi_m$ are in CNF, then the CNF of $\psi \lor \phi$ can be obtained by computing
                \begin{align*}
                          & (\phi_1 \lor \psi_1) \land \dots \land (\phi_1 \lor \psi_m) \\
                    \land & (\phi_2 \lor \psi_1) \land \dots \land (\phi_2 \lor \psi_m) \\
                          & \vdots \\
                    \land & (\phi_n \lor \psi_1) \land \dots \land (\phi_n \lor \psi_m)
                \end{align*}
                Make sure that, for every \code{cf :: ConjForm}, your functions satisfy the following property: \code{simpConj cf == simpConj \$ cnf \$ conjToForm cf}. Example:
                \begin{verbatim}
let (a, b, c, d) = (Pos (Var "a"), Pos (Var "b"),
  Pos (Var "c"), Pos (Var "d"))
cnf $ (L a :&: L b) :|: (L c :&: L d)
  = [[a, c], [a, d], [b, c], [b, d]]
                \end{verbatim}
        \end{enumerate}

    \item \cite[p. 314]{thompson} In this exercise we have a closer look at type inference in Haskell. The most fundamental part of type inference algorithms for a type system using parametric polymorphism is the process of \textit{unifying} two types $T$ and $U$ into the most general type $V$ such that both $T$ and $U$ are so-called \textit{instances} of type $V$. For example, the type \code{(Int,[Char])} is the result of unifying the types \code{(a,[Char])} and \code{(Int,[b])}. \par
        With \textit{type expressions} we refer to composite types. For example, \code{(a,[Char])} is a type expression. \par
        Remember that type variables are universally quantified. An \textit{instance} of a type is given by replacing a type variable or variables by type expressions. All instances of a type form a set describing all possible interpretations of that type. \par
        A type expression is a \textit{common} instance of two type expressions if it is an instance of each expression. The most general common instance of two expressions is a common instance $mgci$ with the property that every other most common instance is an instance of $mgci$. \par
        The intersection of the sets given by two type expressions is called \textit{unification} of the two, which is the most general common instance of the two type expressions. \par
        We define the datatype
        \begin{verbatim}
    data Type = TypeVar Char | Type String [Type] deriving Eq
        \end{verbatim}
        representing type expressions being either a type variable or a type literal that can have type arguments. For example, the type
        \begin{verbatim}
    Tuple a (List Char)
        \end{verbatim}
        is represented as
        \begin{verbatim}
    Type "Tuple" [TypeVar 'a', Type "List" [Type "Char" []]]
        \end{verbatim}
        \begin{enumerate}
            \item Define an instance of \code{Show} for \code{Type} using the same format as in the given example. \textit{Bonus:} print the types \code{Tuple} and \code{List} in a more Haskell-like format.
            \item Define a function \code{unify :: Type -> Type -> Either String Type} that decides whether two type expressions are unifiable. If they are, \code{unify} should return a most general unifying substitution; if not, it should give some explanation of why the unification fails. You may disregard the case of unifying two type variables.
        \end{enumerate}
\end{enumerate}

\subsection{Abstract Data Types}
\begin{enumerate}
    \item {[sheet 12]} Association lists \code{Eq k => [(k,v)]} are a way of representing maps with keys $k$ and values $v$. In order to prevent users from creating invalid association lists (e.g. containing muliple values for some key), we want to hide the implementation in a module.
        \begin{enumerate}
            \item Define a module \code{AssocList} that only exports a type \code{Map k v} and the following functions:
                \begin{verbatim}
    newtype Map k v = ...
    empty :: Map k v
    insert :: Eq k = > k -> v -> Map k v -> Map k v
    lookup :: Eq k = > k -> Map k v -> Maybe v
    delete :: Eq k = > k -> Map k v -> Map k v
    keys :: Map k v -> [k]
                \end{verbatim}
                Calling \code{insert} with an existing key should replace the associated value. Internally, the maps should be represented using association lists. \par
                \textit{Note:} Prelude also exports a function \code{lookup}. To prevent naming conflicts, you can hide this import using \code{import Prelude hiding (lookup)}.
            \item Define a function \code{invar :: Eq k => Map k v -> Bool} in \code{AssocList} that checks whether the map does not contain duplicate keys. Then define QuickCheck properties in a separate module that check whether \code{invar} is invariant under all functions returning a map.
            \item We next check if our implementation (imported as \code{AL.Map}) behaves correctly when compared to the \code{Map} datatype provided by \code{Data.Map} from the package \code{containers} (imported as \code{DM.Map}). Define a function \code{hom :: Ord k => AL.Map k v -> DM.Map k v} that transforms our maps to the one provided by the containers library. Then check whether \code{AL.Map} simulates \code{DM.Map} by defining QuickCheck properties for every function in \code{AssocList}.
        \end{enumerate}
\end{enumerate}

\subsection{Type inference}
\begin{enumerate}
    \item {[endterm 2013]} Determine the most general type of these expressions:
        \begin{enumerate}
            \item \code{filter not}
            \item \code{[] : []}
            \item \code{\textbackslash f x y -> f (x,y)}
            \item \code{map (map fst)}
        \end{enumerate}

    \item {[endterm 2013]} Give a brief justification why the following expression does not type- check.
        \begin{verbatim}
    map head [True, False]
        \end{verbatim}

    \item \cite[p. 319]{thompson} What are the types of
        \begin{enumerate}
            \item \code{curry id}
            \item \code{uncurry id}
            \item \code{curry (curry id)}
            \item \code{uncurry (uncurry id)}
            \item \code{uncurry curry}
        \end{enumerate}

    \item \cite[p. 319]{thompson} Explain why the following expressions do not type-check:
        \begin{enumerate}
            \item \code{curry uncurry}
            \item \code{curry curry}
        \end{enumerate}

    \item {[endterm 2020]} Determine the most general type of these expressions:
        \begin{enumerate}
            \item \code{foldr (\textbackslash x y -> y ++ x) []} (where \code{foldr :: (a -> b -> b) -> b -> [a] -> b})
            \item \code{(\textbackslash f g x -> g \$ f \$ x)}
            \item \code{(:[1,2])}
            \item \code{map head . map (\textbackslash f -> f "hello")}
        \end{enumerate}
\end{enumerate}

\printbibliography

\end{document}
